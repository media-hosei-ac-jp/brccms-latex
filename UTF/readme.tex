%% v1.0 [2022/02/03]
\documentclass{brccms-hu}
%\documentclass[english]{brccms-hu}
\usepackage{amsmath}
%\usepackage{amsthm}
\usepackage[defaultsups]{newtxtext}
\usepackage[varg]{newtxmath}
\usepackage[dvipdfmx]{graphicx}
\usepackage{xcolor}
\usepackage{url}

\def\ClassFile{\texttt{brccms-hu.cls}}

\begin{document}
%\Vol{37}
\jtitle{「法政大学情報メディア教育研究センター研究報告」のための\\
\LaTeXe\ クラスファイル(\ClassFile{})の使い方}
%\jsubtitle{}
\etitle{How to use \ClassFile\ for the Bulletin of Research Center for Computing and Multimedia Studies (RCCMS) of Hosei University}
%\esubtitle{}
\makeatletter
\if@english
\makeatother
\authorlist{%
 \authorentry{First A. Author}{hu}
 \authorentry{Second B. Author}{hue}
}
\affiliate[hu]{Department of ..., University of Hosei University, 
 \Email{000000.ac.jp}}
\affiliate[hue]{Research Center for Computing and Multimedia Studies, 
 Hosei University, \Email{000001.ac.jp}}
\else
\authorlist{%
 \authorentry{第一 著者}{First A. Author}{hu}
 \authorentry{第二 著者}{Second B. Author}{hue}
}
\affiliate[hu]{法政大学○○学部△△学科,\Email{e-mailアドレス}}
\affiliate[hue]{法政大学情報メディア教育研究センター,\Email{e-mailアドレス}}
\fi
%\Jbreakauthorline{4}
%\breakauthorline{4}
%\received{2021}{10}{18}
%\published{2022}{1}{1}

\begin{abstract}
The Research Center for Computing and Multimedia Studies (RCCMS) 
of Hosei University 
provides a (u)p\LaTeXe\ class file, named \ClassFile, 
for the the Bulletin of RCCMS of Hosei University. 
This document describes how to use the class file. 
\end{abstract}
\begin{keyword}
class file, p\LaTeXe, up\LaTeXe
\end{keyword}
\maketitle

\section{はじめに}

このドキュメントは,
「法政大学情報メディア教育研究センター研究報告」
(以下,「研究センター研究報告」と略します)への投稿原稿を,
日本語 (u)p\LaTeXe\ を用いて作成する際に
利用するクラスファイル(\ClassFile{})の使い方を説明したものです.
投稿原稿の執筆にあたっては,『投稿要領』
(https://www.hosei.ac.jp/application/files\slash 8216\slash 0704\slash 
4072\slash bulletin\_howtosubmit.pdf)を参照してください.

本ドキュメントは \LaTeXe\ の基本的な使い方を説明したものではありません.
\LaTeXe\ の使い方に関しては,参考文献の解説書,または
\TeX\ Wiki(https://texwiki.texjp.org/)を参照することを勧めます.

\section{テンプレートならびに記述方法}%\label{template}

\texttt{template-j.tex}(本ドキュメントとともに配布)に沿って
記述すれば,「研究センター研究報告」の体裁を満たします.

\makeatletter
\if@english
\makeatother
\else
\newpage
\fi

\subsection{プリアンブルの記述}

\begin{verbatim}
\documentclass{brccms-hu}
\usepackage{amsmath}
%\usepackage{amsthm}
\usepackage[defaultsups]{newtxtext}
\usepackage[varg]{newtxmath}
\usepackage[dvipdfmx]{graphicx}
%\usepackage[dvips]{graphicx}
\usepackage{xcolor}
\usepackage{url}
\end{verbatim}

\begin{itemize}
\item
今日では \texttt{amsmath} パッケージの利用が一般的です.
\item 
\texttt{amsthm} パッケージを使用するときは,
\texttt{newtxtext} パッケージよりも前に読み込む必要があります
(後ろで読み込むと,そのままではエラーが生じます).
\item 
「研究センター研究報告」では欧文フォントに
\texttt{newtxtext}(タイムス系)を使用します.
\item
\verb/graphicx/ のオプション \texttt{dvipdfmx} は,
ドライバとして \texttt{dvipdfmx} を使うときに指定します.
\texttt{dvips} などの他のドライバを使うときは適宜変更してください.
\item
\texttt{xcolor} もしくは \texttt{color} パッケージは必須ではありません.
\item 
\texttt{url} パッケージは著者のメールアドレスの記述のために読み込みます.
\end{itemize}

\subsection{本文の記述}

最初に和文論文について説明します.
英文論文は\ref{sec:english}項(\pageref{sec:english}頁)で説明します.

\begin{verbatim}
\begin{document}
%\Vol{37}
\jtitle{}
%\jsubtitle{}
\etitle{}
%\esubtitle{}
\authorlist{%
 \authorentry{姓 名}{Mei Sei}{hu}
 \authorentry{姓 名}{Mei Sei}{hue}
}
\affiliate[ラベル]{所属,\Email{メールアドレス}}
%\Jbreakauthorline{4}
%\breakauthorline{4}
%\received{2021}{10}{18}
%\published{2022}{1}{1}

\begin{abstract}
\end{abstract}
\begin{keyword}
\end{keyword}
\maketitle
\section{}
...
\Acknowledgement % 謝辞

\begin{thebibliography}{9}% 文献が10以上のとき99,10未満のとき9など
\bibitem{}
\end{thebibliography}

%\appendix
\end{document}
\end{verbatim}

\begin{itemize}
\item 
\verb/\Vol/ は
巻数をアラビア数字で指定します.
未定のときは,引数を空にするかコメントアウトしてください.

\item 
\verb/\jtitle/ には和文タイトルを記述します.
任意の場所で改行したいときは,
\verb/\\/ か \verb/\break/ を使ってください.
必要に応じて,副題を \verb/\jsubtitle/ コマンドに記述できます.
\item 
\verb/\etitle/ には英文タイトルを記述します.
任意の場所で改行したいときは,\verb/\\/  か \verb/\break/ を使ってください.
必要に応じて,副題を \verb/\esubtitle/ コマンドに記述できます.
\item 
著者名は,以下のように記述します.
\begin{verbatim}
\authorlist{%
 \authorentry{姓 名}{Mei Sei}{ラベル}
}
\end{verbatim}
例えば
\begin{verbatim}
\authorlist{%
 \authorentry{第一 著者}
  {First A. Author}{hu}
}
\end{verbatim}
などと記述します.

\begin{itemize}
\item 
著者のリストを \verb/\authorentry/ に記述し,
リスト全体を \verb/\authorlist/ の引数にします.
\verb/\authorentry/ は何人でも記述できます.
\item
第1引数の和文著者名の
姓と名の間には{\bfseries 必ず ``半角'' のスペース}を挿入します
(スペースを挿入し忘れた場合にはワーニングが出力されます).
\item
第2引数には,著者名のローマ字読みを記述します.
『原稿書式』では「ファーストネーム,ミドルイニシャル,苗字を記載」と
指定されています.
\item
第3引数には,所属のラベルを記述します
(後述の \verb/\affiliate/ の第1引数に対応します).
ラベルの前後にスペースを挿入しないでください.
\verb*/{hu}/ と \verb*/{ hu}/ は異なる所属と判断します.
なお,複数の所属がある場合は,
カンマで区切ってラベルを複数記述することができます.
\item 
著者が多数の場合,任意の場所で改行を行いたいとき,
和名およびローマ字名の場合にそれぞれ,
\verb/\Jbreakauthorline/,
\verb/\breakauthorline/ コマンドを使用します.

例えば,
\verb/\Jbreakauthorline{4}/ とすれば4人目の著者の後ろで改行できます.
\verb/\breakauthorline{4}/ も同様です.
カンマで区切って複数の数字を指定できます.
\end{itemize}

\item 
著者の所属とメールアドレスは \verb/\affiliate/ に記述します.
\begin{verbatim}
\affiliate[ラベル]{所属,\Email{メールアドレス}}
\end{verbatim}
例えば,
\begin{verbatim}
\affiliate[hu]{法政大学○○学部△△学科,\Email{000000.ac.jp}}
\end{verbatim}
などと記述します.

第1引数には \verb/\authorentry/ の第3引数で記述したラベルを
(ラベルの前後にスペースを挿入しないでください),
第2引数には所属・メールアドレスをそれぞれ記述します.
メールアドレスは \verb/\Email/ に,
アドレスをそのまま(例えば \verb/_/ を \verb/\_/ などとしない)
記述してください.
所属が長い場合は \verb/\Email/ の前で \verb/\\/ を使って
改行することができます.

このコマンドは \verb/\authorentry/ で記述したラベルの出現順に記述します.
\item 
\verb/\received/,\verb/\published/ は,
投稿原稿の受付,発行の日付を最初のページの最下部に出力するためのコマンドです.
3つの引数に前から順に,西暦年,月,日のアラビア数字を記述します.
不明の場合は空にするか,コメントアウトしたままにします.
\item
\texttt{abstract} 環境には,英文要旨を記述します.
『原稿書式』では「Abstractの長さは250字以内」と指定されています.
\item
\texttt{keyword} 環境には,英文キーワードを記述します.
『原稿書式』では「キーワードは 6 個以内」と指定されています.
\item 
謝辞を記述する場合は,
\verb/\Acknowledgement/ コマンドを使います.
\end{itemize}

\subsection{英文のテンプレート}
\label{sec:english}

英文用のテンプレートとして \texttt{template-e.tex} を利用してください.

\begin{verbatim}
\documentclass[english]{brccms-hu}
\usepackage{amsmath}
%\usepackage{amsthm}
\usepackage[defaultsups]{newtxtext}
\usepackage[varg]{newtxmath}
\usepackage[dvipdfmx]{graphicx}
%\usepackage[dvips]{graphicx}
\usepackage{xcolor}
\usepackage{url}

\begin{document}
%\Vol{37}
\title{}
%\subtitle{}
\authorlist{%
 \authorentry{First A. Author}{label}
 \authorentry{First B. Author}{hu}
}
\affiliate[label]{affiliation,
 \Email{e-mail address}}
\affiliate[hu]{affiliation, 
 \Email{e-mail address}}
%\breakauthorline{4}
%\received{2021}{10}{18}
%\published{2022}{1}{1}

\begin{abstract}
\end{abstract}
\begin{keyword}
\end{keyword}
\maketitle
\end{verbatim}

和文論文と異なる部分を説明します.

\begin{itemize}
\item 
\verb/\documentclass/ のオプションに \texttt{english} を指定します.
\item 
\verb/\title/ に英文タイトルを記述します.
任意の場所で改行したいときは,
\verb/\\/ か \verb/\break/ を使ってください.
必要に応じて,副題を \verb/\subtitle/ コマンドに記述できます.
\item 
著者名は,引数が2つになり,例えば以下のように記述します.
\begin{verbatim}
\authorlist{%
 \authorentry{First A. Author}{hu}
}
\end{verbatim}
\end{itemize}

\subsection{数式について}
%\label{sec:refs}

別行数式はセンタリングされます.
\begin{align}
 f(x)=\sin x
\end{align}
数式番号は右端に出力されます.

%\makeatletter
%\if@english
%\makeatother
%\else
%\newpage
%\fi

\subsection{図表について}

和文キャプションに加えて英文キャプションが必要です.
英文キャプションは \verb/\ecaption/ に指定します.
\begin{verbatim}
\begin{figure}[htb]
\centering
% graphic etc.
\caption{}
\ecaption{}
\end{figure}
\end{verbatim}

\begin{figure}[htb]
\centering
\textcolor{black!20}{\rule{50mm}{30mm}}
\caption{和文キャプション}
\ecaption{Caption in English}
\end{figure}

表のキャプションは以下のように表組みの上に記述します.
\begin{verbatim}
\begin{table}[htb]
\caption{}
\ecaption{}
\centering
\begin{tabular}{ll}
\hline
.... \\
\hline
\end{tabular}
\end{table}
\end{verbatim}

%\subsection{参考文献について}
%\label{sec:refs}

\subsection{脚注について}
\label{sec:footnote}

脚注%
\footnote{脚注はこのような形で最下段に置かれます.}
は,\LaTeXe\ の標準の形です.

\setbox0\hbox{\verb/\flushbottom/}
\subsection{\box0 について}

\LaTeX\ では二段組のときには \verb/\flushbottom/,
つまり,左右の段の下を揃えるという仕様になっています.
このため,図表や数式などの上下に比較的大きな空きが
生じることがあります.このような場合は,
改段を促すために,適宜 \verb/\newpage/ を使う必要があります.

\subsection{hyperrefについて}
\label{sec:hyperref}

hyperrefを使用するときは,PXjahyperパッケージを併用することを勧めます.
\begin{verbatim}
\usepackage[dvipdfmx]{hyperref}
\usepackage{pxjahyper}
\end{verbatim}

このパッケージが使えないときは,
\texttt{hyperref}オプションに\texttt{setpagesize=false}を
指定することを勧めます.
\begin{verbatim}
\usepackage[dvipdfmx,setpagesize=false]
 {hyperref}
\end{verbatim}

他のパッケージとの併用で生じる不具合などについては,
以下のURLを参照するなどしてください.\par
\begin{small}
\begin{verbatim}
https://texwiki.texjp.org/?hyperref#v71488f4
\end{verbatim}
\end{small}

\section{クラスファイルから削除したコマンド}
%\label{sec:app}

このクラスファイルは「研究センター研究報告」に特化したものです.
目次や索引など使うことのないコマンドは削除しています.

%\Acknowledgement % 謝辞

\begin{thebibliography}{9}% 文献が10以上のとき99,10未満のとき9
\bibitem{Okumura}
奥村晴彦,黒木裕介,
[改訂第8版]\LaTeXe{} 美文書作成入門,
技術評論社,2020.

\bibitem{}
D.E.クヌース,\TeX{} ブック,アスキー出版局,1989.

\bibitem{latex}
レスリー・ランポート,
文書処理システム\LaTeXe{},アスキー出版局,1999.

\bibitem{FMi1}
マイケル・グーセンス,フランク・ミッテルバッハ,アレキサンダー・サマリン,
The \LaTeX{} コンパニオン,アスキー出版局,1998.

\bibitem{FMi2}
マイケル・グーセンス,セバスチャン・ラッツ,フランク・ミッテルバッハ,
\LaTeX{} グラフィックスコンパニオン,アスキー出版局,2000.

\bibitem{PEn}
ページ・エンタープライゼス,
\LaTeXe\ マクロ \& クラスプログラミング基礎解説,
技術評論社,2002.

\bibitem{Yoshinaga}
吉永徹美,
\LaTeXe\ マクロ \& クラスプログラミング実践解説,
技術評論社,2003.
\end{thebibliography}

%\appendix

\end{document}
