%% v1.1 [2023/09/01]
\documentclass{brccms-hu}
\usepackage{amsmath}
%\usepackage{amsthm}% newtxtext よりも前に読み込む
\usepackage[defaultsups]{newtxtext}
\usepackage[varg]{newtxmath}
\usepackage[dvipdfmx]{graphicx}
%\usepackage[dvips]{graphicx}
\usepackage{xcolor}
\usepackage{url}
\usepackage{endnotes}% <- \footnote を使う場合に必要
\usepackage{cite}
\renewcommand*{\citedash}{$-$\penalty\citepunctpenalty}

\begin{document}
%\Vol{37}
\jtitle{「法政大学情報メディア教育研究センター研究報告」原稿書式}
\jsubtitle{―副題―}
\etitle{Instructions for Preparing the Manuscript for the Bulletin of Research Center for Computing and Multimedia Studies (RCCMS) Hosei University}
\esubtitle{-subtitle-}
\authorlist{%
 \authorentry{第一著者}{First A. Author}{hu}
 \authorentry{第二著者}{Second B. Author}{hue}
}
\affiliate[hu]{法政大学○○学部△△学科,\Email{e-mail address}}
\affiliate[hue]{法政大学情報メディア教育研究センター,\Email{e-mail address}}
%\Jbreakauthorline{3}
%\breakauthorline{4}
%\received{2021}{10}{18}
%\published{2022}{1}{1}

\begin{abstract}
The abstract should be concise and contain an explicit summary of your research that states the problem, the methods used, and the major results and conclusions. It should be single-spaced in 9-point Times New Roman. Be sure to adhere to the word limitation for the abstract (250 words). It is advised to avoid referencing in the abstract (unless it is necessary). Please prepare your manuscript in a Microsoft Word file following the specific guidelines provided by Research Center for Computing and Multimedia Studies.
\end{abstract}
\begin{keyword}
Research Report, Personal Computer, Word (Please write no more than six keywords.)
\end{keyword}
\maketitle
\section{はじめに}
この文書は,「法政大学情報メディア教育研究センター研究報告」の紀要に掲載する論文の作成要領を記したものです.原稿はこの要領に沿って慎重に作成してください.このガイドラインに沿ってフォーマットされた論文のみが受理されます.このMicrosoft Wordのテンプレートファイルは,「法政大学情報メディア教育研究センター研究報告」のウェブサイト(https://www.hosei.ac.jp/media/publication/)で入手できます.
このテンプレートは,2022年2月1日版です.
\section{レイアウトとフォント}
\subsection{レイアウト}
データファイルの規定用紙サイズはA4(21cm $\times$29.7cm)です.余白は,上3cm,下3cm,左右各2.5cmです.\LaTeX{}とPDF変換ソフトの設定を確認してください.テキストはすべてシングルスペースで入力してください.以下の2つの例外を除いて,規定の長さの範囲内でご利用ください.(i) 新しいセクションをページの一番下から始めず,見出しを次のページの一番上に移す.(ii) テキストのセクションまたはパラグラフを完成させるために,テキストエリアの長さを1行だけ超えることができます.
\subsection{フォント}
本文のフォントはMS明朝,サイズは10ptです.

AbstractのフォントはTimes-Romanサイズは9ptです.Abstractの長さは250字以内とします.Abstractの中での参照や引用は避けてください.
\subsection{序文}
タイトルラインの前のヘッドラインは変更し
ないでください.また,フッターも変更しないで
ください.最初に論文のタイトルを書いてくださ
い.日本語タイトル,次に英文タイトルを中央に
書き,14pt, MS ゴシック(日),Times-Roman(英)
のフォントを使用してください.タイトルが1 行
以上になる場合は,シングルスペースにしてくだ
さい.

著者名は,日英併記で日本語は姓名,英文はフ
ァーストネーム,ミドルイニシャル,苗字を記載
します,タイトルの2 行下に12pt のMS 明朝(日),
Times-Roman(英)フォントで中央に記入してくだ
さい.著者の所属は,中央揃えで,9pt,MS 明朝
(日),Times-Roman(英)イタリックフォントで,
著者リストの1 行下に記入します.

キーワードは6 個以内とします.キーワードは
左揃えで,9pt,Times-Roman フォントを使用し,
行頭に "Keywords: "と記入してください.それ
ぞれのキーワードはカンマで区切ってください.
キーワードとサマリーの間は1 行空けてくださ
い.
\section{見出し}
見出しは,章,節,小節に対応して,最大で3
段階に分けてください.大見出しは左揃えで,
10.5pt MS ゴシックフォントを使用し,「2.数値
例」のように章番号を前に記載します.大見出し
の前に1 行,後に0 行のスペースを空けてくださ
い.
\subsection{第 2 階層見出し}
第 2 階層の見出しは左揃えで,10pt の MS ゴシ
ックフォントを使用し,本文は MS 明朝フォント
10pt を使用する.見出しの前には 1 行,後には 0
行のスペースを空ける必要があります.
\subsubsection{第 3 階層見出し}
第 3 階層の見出しは,10pt MS 明朝フォント
を使用し,本文は MS 明朝フォント 10pt を使用す
る.「3.1.1 」のようにサブセクション番号を前
に置く.第 3 階層の見出しの前には 1 行,後には
0 行のスペースを空ける必要があります.
\section{数式}
表示される数式には,アラビア数字を括弧で囲
んで番号を付けてください.また,中央に配置し,
上下に 0.5 行のスペースを空けて周囲のテキス
トと区別してください.

式は,タイプライターで書くか,数学エディタ
ーで挿入する必要があります.グラフ,図,アニ
メーションなどの形式で方程式を挿入すること
は避けてください.

次の例は,1 行の方程式です.

次の例は,複数行の方程式です.
\section{表と図}
表と図は、原本または鮮明な印刷物とする。

また、本文中に配置し,最初に取り上げたペー
ジに掲載することが望ましい.

すべての表には連続した番号を付け,キャプシ
ョンを付ける.表は原稿の中央に置き,表のキャ
プションを上にして記載する(表 1).キャプショ
ンのタイトルは中央に書き,10pt MS 明朝フォン
トで,最初に大文字を使用する.表とキャプショ
ンの間には 0.5 行のスペースをとり,キャプショ
ンと表の下部と周囲の文章の間には 1 行のスペ
ースをとる.

また,すべての図や写真には連続した番号を付
け,参照する箇所の直後に本文中に配置する.図
は,原稿の中央に配置し,その下に図のキャプシ
ョンを記載する(図 1).キャプションのタイトル
は中央に書き,10pt MS 明朝フォントを使用し,
最初に大文字を使用します.図とキャプションの
間には 0.5 行のスペースをとり,図の上部とキャ
プションの下部は周囲のテキストと 1 行のスペ
ースをとる.

Web サイトに掲載する電子ファイルは,カラー
画像の掲載を推奨します.表記や文字の高さは
2.5mm 以上とします.
\section{まとめ}
「法政大学情報メディア教育研究センター研
究報告」の紀要は,こちらのファイルから作成さ
れます.そのため,原稿はスペルミスやタイプミ
スがないようにお願いします.投稿後の論文の変
更は認められません.これらの指示に従って作成
されていない論文は審査されず,著者に返却され
ます.
\section{脚注}
脚注は列末とし片段とします。フォントは8pt
の MS 明朝で記載下さい\footnote{脚注は列末とし片段とします.フォントは 8pt の MS
明朝で記載下さい.}.
\begin{table}[htb]
\caption{}
\ecaption{}
\centering
\begin{tabular}{ll}
\hline
%
\hline
\end{tabular}
\end{table}

\begin{figure}[htb]
\centering
% graphic
\caption{}
\ecaption{}
\end{figure}

\Acknowledgement % 謝辞
謝辞(ある場合)は,結論の後に記載してくだ
さい.
\begin{thebibliography}{9}% 文献が10以上のとき99,10未満のとき9など
%引用文献は,雑誌記事[1]および書籍[2]の "参考文献 "の項の例に従ってください.これらは,SIST 02[3]に基づいている.
%(1) 参考文献には,本文中で引用された順に番号をつける [1].
%(2) すべての参考文献は,以下の例のように紙面の最後にまとめて記載する [1].
%(3) 参考文献は,シングルスペース,両端揃え,9pt MS 明朝を使用し,1 列で書いてください.
%(4) 2 つ以上の引用(連続した出典の場合)は,[1][2][3][4]ではなく,[1]-[4]のように引用してください.
%例は以下の通りです
\bibitem{1} 著者名 1, 著者名 2. 書籍タイトル. 出版社,
出版年, 123p.
\bibitem{2} 山本彦文, 田中豊, 新井和吉, 鈴木隆司. 
気泡除去装置内旋回流れの数値解析(気泡除去
性能の評価). 法政大学計算科学研究センター
研究報告. 1999, vol.12, p.1-5.
\bibitem{3} “参照文献の書き方 SIST 02-2007”. 科学
技術振興機構 . https://jipsti.jst.go.jp/sist/pdf/SIST02-
2007.pdf, (accessed 2020-06-18)

\end{thebibliography}

\appendix
付録(ある場合)は,謝辞と参考文献リストの後に記載してください.

\end{document}


%% 2. 英文
\documentclass[english]{brccms-hu}
\usepackage{amsmath}
%\usepackage{amsthm}% newtxtext よりも前に読み込む
\usepackage[defaultsups]{newtxtext}
\usepackage[varg]{newtxmath}
\usepackage[dvipdfmx]{graphicx}
%\usepackage[dvips]{graphicx}
\usepackage{xcolor}
\usepackage{url}
\usepackage{endnotes}% <- \footnote を使う場合に必要

\begin{document}
%\Vol{37}
\title{}
%\subtitle{}
\authorlist{%
 \authorentry{Fist SecondName}{hu}
% \authorentry{Fist SecondName}{hue}
}
\affiliate[hu]{xxx, Hosei University\\
 3--7--2, Kajino-cho, Koganei-shi, Tokyo \jipcode{184--8584}}{e-mail address}
%\affiliate[hue]{xxx, Hosei University\\
% 3--7--2, Kajino-cho, Koganei-shi, Tokyo \jipcode{184--8584}}{e-mail address}

%\breakauthorline{4}
%\received{2021}{10}{18}
%\published{2022}{1}{1}

\begin{abstract}
\end{abstract}
\begin{keyword}
\end{keyword}
\maketitle

\section{}

\begin{table}[htb]
\caption{}
\ecaption{}
\centering
\begin{tabular}{ll}
\hline
%
\hline
\end{tabular}
\end{table}

\begin{figure}[htb]
\centering
% graphic
\caption{}
\ecaption{}
\end{figure}

\Acknowledgement % Acknowledgement

\begin{thebibliography}{9}% 文献が10以上のとき99,10未満のとき9など
\bibitem{}
\end{thebibliography}

\appendix

\end{document}
